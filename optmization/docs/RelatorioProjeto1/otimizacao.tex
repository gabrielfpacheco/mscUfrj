\documentclass[a4paper,12pt,utf8x,notitlepage]{article}

%%%%%%%%%%%%%%%%
%                              		  %
%  Packages and page geometry  %             
%                              		  %             
%%%%%%%%%%%%%%%%

%------------------------------------> Setting language
\usepackage[english, brazil]{babel}
\usepackage[utf8]{inputenc}
\usepackage[T1]{fontenc}
\usepackage{subfigure}
\usepackage{enumerate}

%------------------------------------> To enrich layouts 
\usepackage{amssymb,amsmath,amsthm} %  equations
\usepackage[ruled,vlined]{algorithm2e}     %  algorithms
\usepackage{minitoc, footmisc} 		     %  pages

%------------------------------------> To enrich code layout 
\usepackage{helvet, courier, type1cm}
\usepackage{listings}                   

%------------------------------------> To enrich figures
\usepackage{graphicx}
\usepackage[small]{caption} % for figures legends   
\graphicspath{{./figuras/}} % Default path

\usepackage{epstopdf}
\usepackage{pdfpages}

%------------------------------------> Defining colors
\usepackage{color} 
\definecolor{blackgreen}{rgb}{0,0.4,0}
\definecolor{gray}{gray}{0.3}
%\definecolor{blue}{rgb}{0,0,0.5}
\definecolor{green}{rgb}{0,0.5,0}
\definecolor{lightgreen}{rgb}{0,0.6,0}
\definecolor{purple}{rgb}{0.5,0,0.5}
\definecolor{darkred}{rgb}{0.5,0,0}
\definecolor{dkgreen}{rgb}{0,0.6,0}
\definecolor{mauve}{rgb}{0.58,0,0.82}
\definecolor{orange}{rgb}{1,0.5,0}


%------------------------------------> To enrich annexes 
\usepackage{appendix}

%------------------------------------> Changing default geometry layout
\usepackage{geometry}
\usepackage{setspace}
\addtolength{\hoffset}{-0.3cm}
\addtolength{\voffset}{-.5cm}
\addtolength{\textheight}{1.5cm}
\addtolength{\textwidth}{0.8cm}

%------------------------------------> Using fancy features for header and footpage 
\usepackage{fancyhdr}

%------------------------------------> To create hyperlinks
\usepackage[colorlinks=true, linkcolor=blue, citecolor=blue, urlcolor=blue]{hyperref} %, breaklinks=true


% Para usar matlab2tickz
\usepackage{pgfplots}
% and optionally (as of Pgfplots 1.3):
\pgfplotsset{compat=newest}
\pgfplotsset{plot coordinates/math parser=false}
%\RequirePackage[demo]{graphicx} 

%%%%%%%%%%%%%%%%%%%%%%%%%%%%%%%%%
%                               %
%  Some commands/configuration  %             
%                               %             
%%%%%%%%%%%%%%%%%%%%%%%%%%%%%%%%%

% Formating code typeset structure
\lstset{language=MATLAB, %C
%
basicstyle=\ttfamily\scriptsize,        % code font size
keywordstyle=\color{red}\textbf,	% key words style
commentstyle=\color{blue},		% comments style
%
% numbers=left,                   	% where to put line numbers
% numberstyle=\ttfamily\scriptsize, 	% line nuimbers style
% numberblanklines=false,			
%
captionpos=t,                   % put legend under the text
frame=TB,	                 
%
showspaces=false,               % show spaces
showstringspaces=false,         % show characters
showtabs=false,                 % show tabulation with characters
tabsize=4,	                % tabulation size
%
moredelim=[s][\color{blackgreen}]{'}{'},
moredelim=[s][\color{blackgreen}]{"}{"},
}

%%%%%%%%%%%%%%%%%%%%%%%%
%                      %
%  Personal hot keys   %
%                      %
%%%%%%%%%%%%%%%%%%%%%%%%

\newcommand{\tens}[1]{#1}
\providecommand{\vect}[1]{\mbox{\boldmath${#1}$}}%$
\newcommand{\argmin}[1]{\underset{#1}{\operatorename{argmin}}}
\newcommand{\diag}{\mathop{\mathrm{diag}}}
\newcommand{\norm}[1]{\ensuremath{\left\lVert #1 \right\rVert}}

\newcommand{\TODO}[1]{\textcolor{red}{\large TODO:}\textcolor{purple}{[#1]~}}
\newcommand{\HRule}{\rule{\linewidth}{0.5mm}}
\newcommand{\Lagr}{\mathcal{L}}

\renewcommand{\frame}[1]{\ensuremath{\Psi_{#1}}}        %frame \Psi_{\uppercase{#1}}
\newcommand{\R}[1]{\ensuremath{\mathbb{R}^{#1}}}        %R for real
\newcommand{\Id}[1]{\ensuremath{\tens{I}_{#1}}}         %Identity matrix
\newcommand{\tp}{\ensuremath{^{\mathsf{T}}}}		%transpose
\newcommand{\ft}[2]{\ensuremath{_{#1,#2}}} 		%from to
\newcommand{\rt}[1]{\ensuremath{^{#1}}}			%relative to	transpose
\newcommand{\HM}{\ensuremath{\tens{H}}}			%homogenous matrix
\newcommand{\Rot}{\ensuremath{\tens{R}}}		%rotation matrix
\newcommand{\homo}[1]{\ensuremath{\widetilde{#1}}}	%homogeneous coordinate
\renewcommand{\skew}[1]{\ensuremath{\widehat{#1}}}	%skew-matrix related to a vector
\newcommand{\pt}[1][p]{\ensuremath{\vect{#1}}}		%point in space
\newcommand{\ve}[1][u]{\ensuremath{\vect{#1}}}		%vector in space
\newcommand{\force}{\ensuremath{\vect{f}}}		%force
\newcommand{\torque}{\ensuremath{\vect{\tau}}}		%torque
\newcommand{\ttorque}{\ensuremath{\tau}}		%torque for subscript

\newcommand{\bmat}[1]{\mbox{\boldmath{$#1$}}} 

%%%%%%%%%%%%%%%%%%%%%%%
%                                                                             %
%             		Report Body                             %               
%                                                                             %
%%%%%%%%%%%%%%%%%%%%%%%

\usepackage[]{mcode}

\usepackage{tikz}
\usetikzlibrary{shapes,arrows}
\usetikzlibrary{calc}
\usepackage{verbatim}


\begin{document}
%-------------------------  Gestion des tables des matières et numérotations  -------------------------%
%\frontmatter
\setcounter{tocdepth}{3} 
\setcounter{secnumdepth}{3}

%------------------------------------> Title Page
\begin{titlepage} 
%----------------------------------------------------------------------------
% cover
%----------------------------------------------------------------------------
\thispagestyle{empty}
%\addtolength{\hoffset}{-.5cm}
%\addtolength{\textwidth}{.5cm}
%\addtolength{\voffset}{-.5cm}
%\addtolength{\textheight}{1cm}
\begin{center}
%{\large \bf Universidade Federal do Rio de Janeiro\\}
%\vspace{5pt}
%{\large \bf Escola Politécnica\\}
%\vspace{5pt}
%{\bf  Engenharia de Controle e Automação\\}
\begin{figure}[h!]
	\includegraphics[width=0.45\textwidth]{ufrj.jpg}
	\hfill
	\includegraphics[width=0.32\textwidth]{pee.png}
\end{figure}
\vfill
{\Huge CPE-737 Otimização \\ {\huge Aspectos Teóricos e Métodos Numéricos}}
\vspace{15pt}
\vfill
\rule[2mm]{150mm}{0.2mm}\\
{\bf {\huge Projeto \#2 \\ }
\vspace{0.2cm}
{\Large Métodos numéricos para \\minimização vetorial\\ }
\vspace{0.1cm}}
\rule[-2mm]{150mm}{0.2mm}\\
\vfill
trabalho realizado por \\
\vspace{15pt}
{\large  Arthur {\sc dos Santos Xaud}} \\
{\large  Carolina {\sc Calvo Pose Santos Neves}} \\
{\large  Gabriel {\sc Felippe da Cruz Pacheco}}
\vfill
{ Prof. Afonso {\sc Del Nero Gomes} \\ \vspace{5pt} \today}\\
%\date
\end{center}

\end{titlepage}

% Page's header
\fancyhead{}
\renewcommand{\footrulewidth}{0pt}
\renewcommand{\headrulewidth}{0.4pt}
\pagenumbering{roman}
\setcounter{page}{1}
\pagestyle{fancy}
\fancyhead[LE,RO]{\slshape Projeto \#1}
\fancyhead[LO,RE]{\slshape CPE-737 Otimização}
\fancyfoot{}



%------------------------------------> Table of contents
\tableofcontents
\clearpage

%------------------------------------> List of Figures
%\listoffigures
%\addcontentsline{toc}{chapter}{\listfigurename}



% Page's header
\fancyhead{}
\pagenumbering{arabic}
\setcounter{page}{1}
\fancyhead[LE,RO]{\slshape Projeto \#1}
\fancyhead[LO,RE]{\slshape CPE-737 Otimização}
\renewcommand{\footrulewidth}{0pt}
\renewcommand{\headrulewidth}{0.4pt}
\fancyfoot[C]{\vfill \thepage}

\section{Considerações Iniciais}

\vspace{0.5cm}

Nos algoritmos de otimização apresentados nas seções a seguir (Método de Fibonacci, Seção Áurea e Interpolação Polinomial ), considera-se que as funções de entrada de cada um desses algoritmos são suaves e unimodais no intervalo enquadrante passado como argumento inicialmente. Tais funções são também tais que $\emph{f}:  \mathbb{R} \rightarrow  \mathbb{R}$.\\

Neste relatório, apresenta-se o resultado da convergência em cada um dos três métodos para a mesma função, presente na equação \ref{eq:func_obj}. Outras funções podem ser testadas utilizando a interface gráfica em anexo a este relatório, conforme demonstrado na apresentação do trabalho em sala de aula.

\begin{equation}
f = x^4 + 10 \, x^3
\label{eq:func_obj}
\end{equation}


\subsection{Enquadramento Inicial}

\vspace{0.5cm}

O problema de busca de um mínimo para uma dada função em um determinado intervalo envolve o conceito de unimodalidade de funções. Uma função unimodal em um dado intervalo possui apenas um mínimo naquele intervalo. Dessa maneira, se uma função $\emph{f}$ é unimodal em um intervalo $[a,b]$, então este intervalo enquadra um mínimo e é um intervalo de interesse nos métodos numéricos.\\

Para o método de brent, caso um intervalo seja escolhido, antes da otimização propriamente dita, faz-se um teste para verificar se o intervalo em questão é enquadrante. \\

Neste caso, utilizou-se o critério dos 3 pontos: sejam $x_1, x_2$ e $x_3 \in \mathbb{R}$ tais que $x_1 < x_2 < x_3$ e uma função $\emph{f}:  \mathbb{R} \rightarrow  \mathbb{R}$ suave. Se $\emph{f} (x_1)\geq\emph{f} (x_2)< \emph{f} (x_3)$, então o intervalo $\begin{bmatrix}x_1 & x_3\end{bmatrix}$ enquadra um mínimo. Vale ressaltar que o resultado continua válido se $x_1 > x_2 > x_3$. É importante frizar também que o critério dos três pontos garante que o intervalo é enquadrante mas não necessariamente unimodal, o que significa que pode haver mais de um mínimo neste intervalo.\\

Na situação em que não há um palpite inicial sobre o intervalo alvo para qualquer um dos métodos deste trabalho, procura-se, a partir de $x_1$, $x_2$ e $x_3$ que satisfaçam o critério dos três pontos. Dessa forma, acrescenta-se um $\Delta$ a $x_1$ e uma comparação de $f_2$ e $f_1$ é realizada. Se $f_2 \leq f_1$, procura-se $x_3$ neste mesmo sentido até encontrar $x_3$ tal que $f_3 > f_2$. Caso contrário, caso $f_2>f_1$, inverte-se o sentido da busca. O algoritmo completo do enquadramento inicial por três pontos pode ser facilmente entendido através do diagrama de fluxo da figura \ref{fig:enq}.\\

\begin{figure}[!h]
\centering
\scalebox{.85}{% Define block styles
\tikzstyle{decision} = [diamond, draw, fill=red!20, 
    text width=4em, text centered, node distance=2.5cm, inner sep=0pt]
\tikzstyle{block} = [rectangle, draw, fill=blue!20, 
    text width=6em, text centered, rounded corners, minimum height=3em]
\tikzstyle{block2} = [rectangle, draw, fill=cyan!20, 
    text width=6.5em, text centered,rounded corners, minimum height=3em]
\tikzstyle{line} = [draw, -latex']
\tikzstyle{cloud} = [draw, ellipse,fill=green!20, node distance=2cm,
    minimum height=2em]
    \tikzstyle{io} = [trapezium,trapezium left angle=70,trapezium right angle=-70,minimum height=1.5cm, draw, fill=yellow!20, text width=9em, text badly centered, node distance=2.5cm, inner sep=0pt]
    \tikzstyle{io2} = [trapezium,trapezium left angle=70,trapezium right angle=-70,minimum height=1.5cm, draw, fill=yellow!20, text width=3em, text badly centered, node distance=2.5cm, inner sep=0pt]
    
\begin{tikzpicture}[node distance = 2cm, auto]
    % Place nodes
    \node [io] (init) {$x_1  \in \mathbb{R}$: inicial \\ $\Delta  \in \mathbb{R}$: avanço fixo \\$\gamma \geq 1$ : expans\~ao};
    \node [block,below of = init] (init2) {$x_2 = x_1 + \Delta$ \\ $f_1 = f(x_1)$ \\ $f_2 = f(x_2)$};
    \node [decision, below of=init2] (evaluate1) {$f_2 \leq f_1$};
    \node [block2, left of=evaluate1,node distance=3.5cm] (inversao) {Invers\~ao de sentido: $\Delta = - \Delta$};
    \node [block2,below of=inversao] (passo4) {$a = x_1$, $b = x_2$ \\ $x_1=b$, $x_2 = a$};
    \node [block, below of=evaluate1] (passo5) {$x_3 = x_2 + \gamma \Delta$\\$f_3=f(x_3)$};
    \node [decision, below of=passo5] (evaluate2) {$f_3>f_2$};
    \node [block2, right of = evaluate2,node distance=3.5cm] (passo7) {$x_1 = x_2$ $x_2 = x_3$};
    \node [io2,below of = evaluate2] (passo8) {$a = x_1$ \\ $b = x_3$};
    \node[cloud, below of = passo8](stop){FIM};
    
    % Draw edges
    \path [line] (init) -- (init2);
    \path [line] (init2) -- (evaluate1);
    \path [line] (evaluate1) -- node [near start] {não} (inversao);
    \path [line] (inversao) -- (passo4);
    \path [line] (passo4) -- (passo5);
    \path [line] (evaluate1) -- node [near start] {sim} (passo5);
    \path [line] (passo5) -- (evaluate2);
    \path [line] (evaluate2) -- node [near start] {sim} (passo8);
    \path [line] (evaluate2) -- node [near start] {não} (passo7);
    \path [line,rounded corners] (passo7) |- ($(passo7.north)$) |- (passo5);
    \path[line] (passo8) -- (stop);
 \end{tikzpicture}}
\caption{Enquadramento Inicial por três pontos}
\label{fig:enq}
\end{figure}

\section{Método de Fibonacci}

\vspace{0.5cm}

\subsection{Algoritmo}

Neste método, dado um intervalo enquadrante $I_k = [a \,  b]$, no qual a função $f$ é unimodal, escolhem-se dois pontos internos simétricos em relação ao centro, conforme a figura \ref{fig:fib}.

\begin{figure}[!h]
\centering
    \includegraphics[width=0.35\textwidth]{fibonacci.pdf}
    \caption{Pontos do método de Fibonnaci}
   \label{fig:fib}
\end{figure}

Um fator de redução inicial $0.5<\alpha_1<0.7$ é usado para encontrar os primeiros valores de $x_2$ e $x_3$ dentro do intervalo enquadrante inicial $I_1 = [a \, b] \subset \mathbb{R}$. \\

Então, a função-objetivo é avaliada nesses dois pontos $x_2$ e $x_3$ e o intervalo é reduzido de acordo com a comparação dos valores de $f(x_2)= f_2$ e $f(x_3)=f_3$: se $f_2<f_3$, o novo intervalo é definido como $I_{k+1}=[x_1 \,x_3]$. Caso contrário, o intervalo seguinte passa a ser $I_{k+1}=[x_2 \,x_4]$. Finalmente, o fator de redução é recalculado de acordo com a equação \ref{eq:alpha} e o procedimento se repete até que o número de reduções desejado $n \in \mathbb{Z}$ seja atingido.

\begin{equation}
\alpha = \frac{L_{ini}-L_{fim}}{L_{fim}}
\label{eq:alpha}
\end{equation}

Novamente, o diagrama de fluxo do algoritmo implementado é apresentado na figura \ref{fig:fibonacci}.

\begin{figure}[h!]
\centering
\scalebox{0.85}{% Define block styles
\tikzstyle{decision} = [diamond, draw, fill=red!20, 
    text width=4.5em, text centered, node distance=2.5cm, inner sep=0pt]
\tikzstyle{block} = [rectangle, draw, fill=blue!20, node distance=2.5cm,
    text width=9em, text centered, rounded corners, minimum height=1em]
\tikzstyle{block2} = [rectangle, draw, fill=cyan!20, 
    text width=6.5em, text centered,rounded corners, minimum height=3em]
\tikzstyle{line} = [draw, -latex']
\tikzstyle{cloud} = [draw, ellipse,fill=green!20, node distance=1.5cm,
    minimum height=2em]
    \tikzstyle{io} = [trapezium,trapezium left angle=70,trapezium right angle=-70,minimum height=2cm, draw, fill=yellow!20, text width=7em, text badly centered, node distance=2.5cm, inner sep=0pt]
    \tikzstyle{io2} = [trapezium,trapezium left angle=70,trapezium right angle=-70,minimum height=0.7cm, draw, fill=yellow!20, text width=4em, text badly centered, node distance=2cm, inner sep=0pt]
    
    
\begin{tikzpicture}[node distance = 2cm, auto]
    % Place nodes
    \node [io] (input) {$I_1: [a\;b]$ \\ $n: \#$ reducoes \\ k : indice de Fibonnacci};
    \node [block,below of = input] (init) {$p = \frac{1-\sqrt{5}}{1+\sqrt{5}};\,k=n+1$\\$\alpha = \frac{2}{1+\sqrt{5}}\,\frac{1-p^k}{1-p^{k+1}}$};
    \node [block,below of = init] (init2) {$x_1 = a; \,x_4 = b$\\$L_{ini} = b-a;\,i=1$\\$x_2 = \alpha x_1 +(1-\alpha)x_4$\\$x_3=\alpha x_4 +(1-\alpha)x_1$};
    \node [decision, below of=init2] (evaluate) {$f_2 < f_3$};
    \node[below of = evaluate](aux){};
    \node [block2, left of=aux,node distance=3.5cm] (passo7) {$a=x_2$\\ $b=x_4$ \\$L_{fim} = b-a$};
    \node [block2, right of=aux,node distance=3.5cm] (passo6) {$a = x_1$\\ $b = x_3$\\ $L_{fim} = b-a$};
    \node [decision,below of=aux] (tolerancia) {$i=n?$};
    %\node[below of = tolerancia](aux2){};
    \node [block2, left of = tolerancia,node distance=3.5cm] (passo67parte2) {$\alpha = \frac{L_{ini}-L_{fim}}{L_{fim}}$\\$i = i+1$};
    \node[io2, below of = tolerancia](out){$I_f:[a\;b]$};
    \node[cloud, below of = out](stop){FIM};
     
        
    % Draw edges
    \path [line] (input) -- (init);
    \path [line] (init) -- (init2);
    \path [line] (init2) -- (evaluate);
    \path [line] (evaluate.south west) -- node [above]{não} (passo7);
    \path [line] (evaluate.south east) -- node [above] {sim} (passo6);
    \path [line] (passo7) -- (tolerancia);
    \path [line] (passo6) -- (tolerancia);
    \path [line] (tolerancia) -- node [near start] {não} (passo67parte2);
    \path[line] (tolerancia) -- node [near start] {sim} (out);
    \path[line] (out) -- (stop);
    \draw [line,rounded corners] (passo67parte2.west) -- ++(-0.5,0) |- node[anchor=west] {} (init2.west);
\end{tikzpicture}}
\caption{Método de Fibonacci}
\label{fig:fibonacci}
\end{figure}

\subsection{Resultado}

O resultado obtido utilizando a função teste enunciada em \ref{eq:func_obj} é apresentado na figura \ref{fig:fibo}

\begin{figure}[!h]
\centering
\scalebox{0.85}{% Define block styles
\tikzstyle{decision} = [diamond, draw, fill=red!20, 
    text width=4.5em, text centered, node distance=2.5cm, inner sep=0pt]
\tikzstyle{block} = [rectangle, draw, fill=blue!20, node distance=2.5cm,
    text width=9em, text centered, rounded corners, minimum height=1em]
\tikzstyle{block2} = [rectangle, draw, fill=cyan!20, 
    text width=6.5em, text centered,rounded corners, minimum height=3em]
\tikzstyle{line} = [draw, -latex']
\tikzstyle{cloud} = [draw, ellipse,fill=green!20, node distance=1.5cm,
    minimum height=2em]
    \tikzstyle{io} = [trapezium,trapezium left angle=70,trapezium right angle=-70,minimum height=2cm, draw, fill=yellow!20, text width=7em, text badly centered, node distance=2.5cm, inner sep=0pt]
    \tikzstyle{io2} = [trapezium,trapezium left angle=70,trapezium right angle=-70,minimum height=0.7cm, draw, fill=yellow!20, text width=4em, text badly centered, node distance=2cm, inner sep=0pt]
    
    
\begin{tikzpicture}[node distance = 2cm, auto]
    % Place nodes
    \node [io] (input) {$I_1: [a\;b]$ \\ $n: \#$ reducoes \\ k : indice de Fibonnacci};
    \node [block,below of = input] (init) {$p = \frac{1-\sqrt{5}}{1+\sqrt{5}};\,k=n+1$\\$\alpha = \frac{2}{1+\sqrt{5}}\,\frac{1-p^k}{1-p^{k+1}}$};
    \node [block,below of = init] (init2) {$x_1 = a; \,x_4 = b$\\$L_{ini} = b-a;\,i=1$\\$x_2 = \alpha x_1 +(1-\alpha)x_4$\\$x_3=\alpha x_4 +(1-\alpha)x_1$};
    \node [decision, below of=init2] (evaluate) {$f_2 < f_3$};
    \node[below of = evaluate](aux){};
    \node [block2, left of=aux,node distance=3.5cm] (passo7) {$a=x_2$\\ $b=x_4$ \\$L_{fim} = b-a$};
    \node [block2, right of=aux,node distance=3.5cm] (passo6) {$a = x_1$\\ $b = x_3$\\ $L_{fim} = b-a$};
    \node [decision,below of=aux] (tolerancia) {$i=n?$};
    %\node[below of = tolerancia](aux2){};
    \node [block2, left of = tolerancia,node distance=3.5cm] (passo67parte2) {$\alpha = \frac{L_{ini}-L_{fim}}{L_{fim}}$\\$i = i+1$};
    \node[io2, below of = tolerancia](out){$I_f:[a\;b]$};
    \node[cloud, below of = out](stop){FIM};
     
        
    % Draw edges
    \path [line] (input) -- (init);
    \path [line] (init) -- (init2);
    \path [line] (init2) -- (evaluate);
    \path [line] (evaluate.south west) -- node [above]{não} (passo7);
    \path [line] (evaluate.south east) -- node [above] {sim} (passo6);
    \path [line] (passo7) -- (tolerancia);
    \path [line] (passo6) -- (tolerancia);
    \path [line] (tolerancia) -- node [near start] {não} (passo67parte2);
    \path[line] (tolerancia) -- node [near start] {sim} (out);
    \path[line] (out) -- (stop);
    \draw [line,rounded corners] (passo67parte2.west) -- ++(-0.5,0) |- node[anchor=west] {} (init2.west);
\end{tikzpicture}}
\caption{Número de iterações: $n=10$. Tamanho do último intervalo $L_f = 0.1215$}
\label{fig:fibo}
\end{figure}

O número de iterações escolhido foi $n=10$ para que o método fosse suficientemente distinto do método da seção áurea. Tal número de iterações nos fornece um intervalo final $L_f = 0.1215$. O resultado poderia ser melhorado arbritariamente apenas aumentando o valor de $n$.

\section{Seção Áurea}

\vspace{0.5cm}

\subsection{Algoritmo}

Este método se baseia na divisão áurea de um segmento. Dado um segmento de reta $\overline{AB}$, dividi-lo na chamada razão áurea significa achar um ponto M deste segmento tal que o segmento todo está para a parte maior assim como esta parte maior está para a parte menor, conforme a figura \ref{fig:div_gold}.

\begin{figure}[!h]
\centering
    \includegraphics[width=0.55\textwidth]{gold.pdf}
    \caption{Divisão áurea de um segmento}
\label{fig:div_gold}
\end{figure}

O valor da razão entre esses segmentos que realiza isso é dada pela equação \ref{eq:u}.
\begin{equation}
u = \frac{\sqrt{5}-1}{2}  \approx 0.6180
\label{eq:u}
\end{equation}

O método da seção áurea é bastante parecido com o método de Fibonacci. A diferença reside no fato de que os pontos centrais $x_2$ e $x_3$ são determinados através da razão áurea da equação \ref{eq:u}. Além disso, nesse caso, não se determina um número desejado de reduções como no método de Fibonacci, mas sim uma tolerância $\varepsilon \in  \mathbb{R}$ para o intervalo final.

De fato, um incoveniente do algoritmo de Fibonacci está relacionado à necessidade de se conhecer a priori o número total de reduções e o índice k associado, ao passo que, no método da divisão áurea, isso não é necessário. Na verdade, demonstra-se inclusive que, se o número de reduções prevista no algoritmo de Fibonacci for grande (n>10), a primeira redução do método de Fibonacci corresponde justamente à divisão áurea.\\

O diagrama de fluxo da implentação do algoritmo da seção áurea é apresentado na figura \ref{fig:gold}.

\begin{figure}[h!]
\centering
\scalebox{.85}{% Define block styles
\tikzstyle{decision} = [diamond, draw, fill=red!20, 
    text width=4.5em, text centered, node distance=3cm, inner sep=0pt]
\tikzstyle{block} = [rectangle, draw, fill=blue!20, 
    text width=9em, text centered, rounded corners, minimum height=1em]
\tikzstyle{block2} = [rectangle, draw, fill=cyan!20, node distance=2.5cm,
    text width=9em, text centered,rounded corners, minimum height=3.5em]
\tikzstyle{line} = [draw, -latex']
\tikzstyle{cloud} = [draw, ellipse,fill=green!20, node distance=2cm,
    minimum height=2em]
    \tikzstyle{io} = [trapezium,trapezium left angle=70,trapezium right angle=-70,minimum height=1cm, draw, fill=yellow!20, text width=5.5em, text badly centered, node distance=2.5cm, inner sep=0pt]
    \tikzstyle{io2} = [trapezium,trapezium left angle=70,trapezium right angle=-70,minimum height=0.5cm, draw, fill=yellow!20, text width=4em, text badly centered, node distance=3.5cm, inner sep=0pt]
    
    
\begin{tikzpicture}[node distance = 2cm, auto]
    % Place nodes
    \node [io] (input) {$I_1: [a\;b]$ \\ $\varepsilon$: toler\^ancia};
    \node [block,below of = input] (init) {$x_1 = a$; $x_4 = b$ \\$L_i = b - a$; $u = \frac{\sqrt{5}-1}{2}$\\$x_2 = u x_1 + (1-u) x_4$\\$x_3 = (1-u)x_1 + ux_4$};
    \node [decision, below of=init] (evaluate) {$f_2 < f_3$};
    \node[below of = evaluate](aux){};
    \node [block2, left of=aux,node distance=3.5cm] (passo7) {$a=x_2$\\ $b=x_4$ \\$L_f = b-a$};
    \node [decision,below of=passo7] (tolerancia) {$L_f < \varepsilon$};
    \node [block2, right of=aux,node distance=3.5cm] (passo6) {$a = x_1$\\ $b = x_3$\\ $L_f = b-a$};
    \node [decision,below of=passo6] (tolerancia2) {$L_f < \varepsilon$};
    \node [block2, below of = tolerancia] (passo7parte2) {$ x_1 = x_2; \,x_2 = x_3$\\$x_3 = (1-u)a+ub$};
    \node [block2, below of = tolerancia2] (passo6parte2) {$x_4 = x_3; \, x_3 = x_2$\\$x_2 = ua+(1-u)b$};
    \node[io2, right of = tolerancia,node distance=3.5cm](out){$I_f:[a\;b]$};
    \node[cloud, below of = out](stop){FIM};
     
        
    % Draw edges
    \path [line] (input) -- (init);
    \path [line] (init) -- (evaluate);
    \path [line] (evaluate.south west) -- node [above]{não} (passo7);
    \path [line] (evaluate.south east) -- node [above] {sim} (passo6);
    \path [line] (passo7) -- (tolerancia);
    \path [line] (passo6) -- (tolerancia2);
    \path [line] (tolerancia) -- node [near start] {não} (passo7parte2);
    \path [line] (tolerancia2) -- node [near start] {não} (passo6parte2);
    \path[line] (tolerancia) -- node [above] {sim} (out);
    \path[line] (tolerancia2) -- node [above] {sim} (out);
    \path[line] (out) -- (stop);
    \draw [line,rounded corners] (passo7parte2.west) -- ++(-0.5,0) |- node[anchor=west] {} (evaluate.west);
    \draw [line,rounded corners] (passo6parte2.east) -- ++(0.5,0) |- node[anchor=east] {} (evaluate.east);
 \end{tikzpicture}}
\caption{Seção Áurea}
\label{fig:gold}
\end{figure}

\subsection{Resultado}

\vspace{0.5cm}

O resultado obtido utilizando a função teste enunciada em \ref{eq:func_obj} é apresentado na figura \ref{fig:aurea}.\\

\begin{figure}[h!]
\centering
\scalebox{.85}{\input{./figuras/secao_aurea.tex}}
\caption{Tolerância: $\epsilon =  0.01$. Número de iterações: $n = 17$}
\label{fig:aurea}
\end{figure}

Para atingir a tolerância estipulada de $\epsilon =  0.01$, foram necessárias 17 iterações do algoritmo. O algoritmo da seção áurea é extremamente robusto e seguro, no entanto, veremos como uma aproximação da função objetivo por um polinômio quadrático mesclado ao algoritmo da seção áurea pode ter uma performance superior.

\section{Interpolação Parabólica e o Método de Brent}

\vspace{0.5cm}

\subsection{Princípios básicos}

A idéia por trás desse método é utilizar uma aproximação da função $f(x)$ por um polinômio de grau 2 $g(x)$ que passa por três pontos pertencentes a um intervalo enquadrante. Dessa maneira, aproxima-se o mínimo de $f(x)$ pelo mínimo da parábola obtida. A fórmula \ref{eq:parab} permite encontrar a abcissa $x$ correspondente ao mínimo da parábola para os pontos $a,b$ e $c$.\\

\begin{equation}
  x = b - \frac{1}{2}\frac{(b-a)^2[f(b)-f(c)] - (b-c)^2[f(b)-f(a)]}{(b-a)[f(b)-f(c)] - (b-c)[f(b)-f(a)]} 
  \label{eq:parab}
\end{equation}

Redefine-se, então, o intervalo para a próxima iteração de acordo com a avaliação da função objetivo nesses três pontos, descartando o pior deles e guardando o minimizador desta parábola aproximada como novo ponto. Tal interpolação parabólica é dita inversa pois buscamos a abscissa $x$ e não a ordenada $f(x)$. \\

Da expressão \ref{eq:parab}, podemos notar que existe uma singularidade quando os três pontos são colineares. Além disso, o método funciona da mesma maneira tanto na presença de um mínimo quanto de um máximo. \\

Por isso, para contornar tais dificuldades, a estratégia proposta por Brent é utilizar uma busca segura (ex: seção-áurea) quando que a função for não cooperativa e trocar para para interpolação parabólica inversa sempre que possível. Assim, incorpora-se a robustez do método da seção-áurea com a rapidez da interpolação quadrática.\\

\begin{figure}[!h]
\centering
    \includegraphics[width=0.55\textwidth]{parabola.pdf}
    \caption{Convergência pelo Método de Brent. Figura retirada de \cite{Brent} }
\label{fig:parabolas}
\end{figure}

A figura \ref{fig:parabolas} mostra duas iterações da interpolação parabólica. É possível notar que a primeira estimativa do minimizador é obtida em $4$ e que, em seguida, esta muda para $5$, se aproximando ainda mais do real valor do minimizador da função $f(x)$.
 
\subsection{Algoritmo}

Durante qualquer fase do algoritmo, este possui um conjunto de 6 pontos $(a, b, u, v, w, x)$. Os pontos $a$ e $b$ definem o intervalo enquadrante de cada iteração; $x$ é o minimizador de $f(x)$ obtido até o momento; $w$ é o ponto de segundo menor valor da função objetivo até o momento e $v$ é o valor anterior de $w$; $u$, por sua vez,  é o ponto onde a função foi avaliada mais recentemente. Um outro ponto que também aparece no algoritmo é o ponto médio entre $a$ e $b$, i.e $x_m$, no entanto, a função objetivo não é avaliada nele.

Se após o passo $d$ da interpolação parabólica, o melhor ponto $x$ pertencer ao intervalo $[a, \, b]$ e, ainda, se o deslocamento provocado por este passo for inferior ao deslocamento provocado na penúltima iteração, significa que $x$ está convergindo e que a interpolação parabólica pode ser usada. Caso alguma dessas condições não seja satisfeita, o passo $d$ é calculado utilizando a razão áurea do maior segmento dentro do intervalo $[a, \,b]$.

\begin{figure}[h!]
\centering
\scalebox{.85}{% Define block styles
\tikzstyle{io} = [trapezium,trapezium left angle=70,trapezium right angle=-70,minimum height=1cm, draw, fill=yellow!20, text width=6em, text badly centered, node distance=2.5cm, inner sep=0pt]
\tikzstyle{decision} = [diamond, draw, fill=red!20, 
    text width=6em, text badly centered, node distance=3cm, inner sep=0pt]
\tikzstyle{block} = [rectangle, draw, fill=blue!20, 
    text width=10em, text centered, rounded corners, minimum height=4em]
    \tikzstyle{block0} = [rectangle, draw, fill=blue!20, 
    text width=4em, text centered, rounded corners, minimum height=4em]
    \tikzstyle{block1} = [rectangle, draw, fill=cyan!20, node distance=3cm,
    text width=5em, text centered, rounded corners, minimum height=4em]
    \tikzstyle{line} = [draw, -latex']
\tikzstyle{cloud} = [draw, ellipse,fill=green!20, node distance=2cm,
    minimum height=2em]
    \tikzstyle{io2} = [trapezium,trapezium left angle=70,trapezium right angle=-70,minimum height=1cm, draw, fill=yellow!20, text width=1.5cm, text badly centered, node distance=3cm, inner sep=0pt]
    
\begin{tikzpicture}[node distance = 2cm, auto]
    % Place nodes
    \node [io] (input) {$I_1 = [a\, b]$\\$\varepsilon: tolerancia$};
    \node [block, below of =input] (init) {Atualiza $x,w,v,u$\\$x_m = \frac{a+b}{2}$}; % $x=w=v$: criterio 3 pontos\\$d_0=0;
    %\node [block, below of = init] (parab) {Atualiza:\\$x$:minimizador parabola\\$w$:segundo menor valor f\\$v$ : $w$ anterior\\$u$: ultima avaliacao $f$\\};
    \node [block, below of =init] (parab) {$r = (x-w)\,(f_x-f_v)$\\$q = (x-v)\,(f_x-f_w)$\\$p = (x-v)\,q - (x-w)\,r$};
    \node [decision, below of=parab] (evaluate) {$(x+d)\in [a\,b]$\\ $\&\&$\\ $d_k<\frac{d_{k-2}}{2}$};
    \node [below of = evaluate](aux){};
    \node [block1,right of = aux,node distance=3cm] (interp) {$d=\frac{p}{q}$};
    \node [block1, left of=aux, node distance=3cm] (gold) {d: Seção Áurea};
    \node [block0, below of=aux] (act) {$u = x + d$};
    \node [decision, below of=act] (evaluate2) {$|x - x_m|<\varepsilon$};
    \node [io2, below of=evaluate2] (out) {$x_{min}=x$};
    \node [cloud,below of = out] (stop) {FIM};
    %\node [left of = parab_step, node distance=3cm](aux2){};
       
    % Draw edges
    \path [line] (input) -- (init);
    \path [line] (init) -- (parab);
    \path [line] (parab) -- (evaluate);
    \path [line,rounded corners] (evaluate.east) -| node [above] {sim} (interp);
    \path [line,rounded corners] (evaluate.west) -| node [above] {não}(gold);
    \path [line,rounded corners] (gold.south) |- (act.west);
    \path [line,rounded corners] (interp.south) |- (act.east);
    \path[line] (act) -- (evaluate2);
    \path [line] (evaluate2) -- node [near start] {sim} (out);
    \path [line,rounded corners] (evaluate2.east)-- ++(35mm,0mm) |- node [above] {não} (init.east);
    \path [line] (out) -- (stop);
\end{tikzpicture}}
\caption{Fluxograma do Método de Interpolação}
\label{fig:brentFlow}
\end{figure}

O algoritmo retoma o início dos cálculos dos novos pontos $(a, b, u, v, w, x)$ até que a tolerância seja atingida ($ |x-x_m| < tol $), alternando,portanto, entre e a interpolação polinomial e o método da razão áurea na determinação dos intervalos para as próximas iterações. O diagrama de fluxo \ref{fig:brentFlow} exemplifica o algoritmo.

\subsection{Resultado}

O resultado obtido para o método de Brent utilizando a função teste apresentada previamente neste relatório (eq. \ref{eq:func_obj}) pode ser observado na figura \ref{fig:brent}.\\

\begin{figure}[h!]
\centering
\scalebox{0.85}{\section{Interpolação Parabólica e o Método de Brent}

\vspace{0.5cm}

\subsection{Princípios básicos}

\vspace{0.5cm}

A idéia por trás desse método é utilizar uma aproximação da função $f(x)$ por um polinômio de grau 2 $g(x)$ que passa por três pontos pertencentes a um intervalo enquadrante. Dessa maneira, aproxima-se o mínimo de $f(x)$ pelo mínimo da parábola obtida. A fórmula \ref{eq:parab} permite encontrar a abcissa $x$ correspondente ao mínimo da parábola para os pontos $a,b$ e $c$.\\

\begin{equation}
  x = b - \frac{1}{2}\frac{(b-a)^2[f(b)-f(c)] - (b-c)^2[f(b)-f(a)]}{(b-a)[f(b)-f(c)] - (b-c)[f(b)-f(a)]} 
  \label{eq:parab}
\end{equation}

\vspace{0.5cm}

Redefine-se, então, o intervalo para a próxima iteração de acordo com a avaliação da função objetivo nesses três pontos, descartando o pior deles e guardando o minimizador desta parábola aproximada como novo ponto. Tal interpolação parabólica é dita inversa pois buscamos a abscissa $x$ e não a ordenada $f(x)$. \\

Da expressão \ref{eq:parab}, podemos notar que existe uma singularidade quando os três pontos são colineares. Além disso, o método funciona da mesma maneira tanto na presença de um mínimo quanto de um máximo. \\

Por isso, para contornar tais dificuldades, a estratégia proposta por Brent é utilizar uma busca segura (ex: seção-áurea) quando que a função for não cooperativa e trocar para para interpolação parabólica inversa sempre que possível. Assim, incorpora-se a robustez do método da seção-áurea com a rapidez da interpolação quadrática.\\

\begin{figure}[!h]
\centering
    \includegraphics[width=0.75\textwidth]{parabola.pdf}
    \caption{Convergência pelo Método de Brent. Figura retirada de \cite{Brent} }
\label{fig:parabolas}
\end{figure}

A figura \ref{fig:parabolas} mostra duas iterações da interpolação parabólica. É possível notar que a primeira estimativa do minimizador é obtida em $4$ e que, em seguida, esta muda para $5$, se aproximando ainda mais do real valor do minimizador da função $f(x)$.
 
\subsection{Algoritmo}

\vspace{0.5cm}

Durante qualquer fase do algoritmo, este possui um conjunto de 6 pontos $(a, b, u, v, w, x)$. Os pontos $a$ e $b$ definem o intervalo enquadrante de cada iteração; $x$ é o minimizador de $f(x)$ obtido até o momento; $w$ é o ponto de segundo menor valor da função objetivo até o momento e $v$ é o valor anterior de $w$; $u$, por sua vez,  é o ponto onde a função foi avaliada mais recentemente. Um outro ponto que também aparece no algoritmo é o ponto médio entre $a$ e $b$, i.e $x_m$, no entanto, a função objetivo não é avaliada nele.

Se após o passo $d$ da interpolação parabólica, o melhor ponto $x$ pertencer ao intervalo $[a, \, b]$ e, ainda, se o deslocamento provocado por este passo for inferior ao deslocamento provocado na penúltima iteração, significa que $x$ está convergindo e que a interpolação parabólica pode ser usada. Caso alguma dessas condições não seja satisfeita, o passo $d$ é calculado utilizando a razão áurea do maior segmento dentro do intervalo $[a, \,b]$.\\

O algoritmo retoma o início dos cálculos dos novos pontos $(a, b, u, v, w, x)$ até que a tolerância seja atingida ($ |x-x_m| < tol $), alternando,portanto, entre e a interpolação polinomial e o método da razão áurea na determinação dos intervalos para as próximas iterações. O diagrama de fluxo \ref{fig:brentFlow} exemplifica o algoritmo.

 \begin{figure}
 \centering
 \scalebox{1}{% Define block styles
\tikzstyle{io} = [trapezium,trapezium left angle=70,trapezium right angle=-70,minimum height=1cm, draw, fill=yellow!20, text width=6em, text badly centered, node distance=2.5cm, inner sep=0pt]
\tikzstyle{decision} = [diamond, draw, fill=red!20, 
    text width=6em, text badly centered, node distance=3cm, inner sep=0pt]
\tikzstyle{block} = [rectangle, draw, fill=blue!20, 
    text width=10em, text centered, rounded corners, minimum height=4em]
    \tikzstyle{block0} = [rectangle, draw, fill=blue!20, 
    text width=4em, text centered, rounded corners, minimum height=4em]
    \tikzstyle{block1} = [rectangle, draw, fill=cyan!20, node distance=3cm,
    text width=5em, text centered, rounded corners, minimum height=4em]
    \tikzstyle{line} = [draw, -latex']
\tikzstyle{cloud} = [draw, ellipse,fill=green!20, node distance=2cm,
    minimum height=2em]
    \tikzstyle{io2} = [trapezium,trapezium left angle=70,trapezium right angle=-70,minimum height=1cm, draw, fill=yellow!20, text width=1.5cm, text badly centered, node distance=3cm, inner sep=0pt]
    
\begin{tikzpicture}[node distance = 2cm, auto]
    % Place nodes
    \node [io] (input) {$I_1 = [a\, b]$\\$\varepsilon: tolerancia$};
    \node [block, below of =input] (init) {Atualiza $x,w,v,u$\\$x_m = \frac{a+b}{2}$}; % $x=w=v$: criterio 3 pontos\\$d_0=0;
    %\node [block, below of = init] (parab) {Atualiza:\\$x$:minimizador parabola\\$w$:segundo menor valor f\\$v$ : $w$ anterior\\$u$: ultima avaliacao $f$\\};
    \node [block, below of =init] (parab) {$r = (x-w)\,(f_x-f_v)$\\$q = (x-v)\,(f_x-f_w)$\\$p = (x-v)\,q - (x-w)\,r$};
    \node [decision, below of=parab] (evaluate) {$(x+d)\in [a\,b]$\\ $\&\&$\\ $d_k<\frac{d_{k-2}}{2}$};
    \node [below of = evaluate](aux){};
    \node [block1,right of = aux,node distance=3cm] (interp) {$d=\frac{p}{q}$};
    \node [block1, left of=aux, node distance=3cm] (gold) {d: Seção Áurea};
    \node [block0, below of=aux] (act) {$u = x + d$};
    \node [decision, below of=act] (evaluate2) {$|x - x_m|<\varepsilon$};
    \node [io2, below of=evaluate2] (out) {$x_{min}=x$};
    \node [cloud,below of = out] (stop) {FIM};
    %\node [left of = parab_step, node distance=3cm](aux2){};
       
    % Draw edges
    \path [line] (input) -- (init);
    \path [line] (init) -- (parab);
    \path [line] (parab) -- (evaluate);
    \path [line,rounded corners] (evaluate.east) -| node [above] {sim} (interp);
    \path [line,rounded corners] (evaluate.west) -| node [above] {não}(gold);
    \path [line,rounded corners] (gold.south) |- (act.west);
    \path [line,rounded corners] (interp.south) |- (act.east);
    \path[line] (act) -- (evaluate2);
    \path [line] (evaluate2) -- node [near start] {sim} (out);
    \path [line,rounded corners] (evaluate2.east)-- ++(35mm,0mm) |- node [above] {não} (init.east);
    \path [line] (out) -- (stop);
\end{tikzpicture}}
 \caption{Fluxograma do Método de Interpolação}
 \label{fig:brentFlow}
 \end{figure}

\subsection{Resultado}

O resultado obtido para o método de Brent utilizando a função teste apresentada previamente neste relatório (eq. \ref{eq:func_obj}) pode ser observado na figura \ref{fig:brent}.\\

\begin{figure}[h!]
\centering
\scalebox{1}{\section{Interpolação Parabólica e o Método de Brent}

\vspace{0.5cm}

\subsection{Princípios básicos}

\vspace{0.5cm}

A idéia por trás desse método é utilizar uma aproximação da função $f(x)$ por um polinômio de grau 2 $g(x)$ que passa por três pontos pertencentes a um intervalo enquadrante. Dessa maneira, aproxima-se o mínimo de $f(x)$ pelo mínimo da parábola obtida. A fórmula \ref{eq:parab} permite encontrar a abcissa $x$ correspondente ao mínimo da parábola para os pontos $a,b$ e $c$.\\

\begin{equation}
  x = b - \frac{1}{2}\frac{(b-a)^2[f(b)-f(c)] - (b-c)^2[f(b)-f(a)]}{(b-a)[f(b)-f(c)] - (b-c)[f(b)-f(a)]} 
  \label{eq:parab}
\end{equation}

\vspace{0.5cm}

Redefine-se, então, o intervalo para a próxima iteração de acordo com a avaliação da função objetivo nesses três pontos, descartando o pior deles e guardando o minimizador desta parábola aproximada como novo ponto. Tal interpolação parabólica é dita inversa pois buscamos a abscissa $x$ e não a ordenada $f(x)$. \\

Da expressão \ref{eq:parab}, podemos notar que existe uma singularidade quando os três pontos são colineares. Além disso, o método funciona da mesma maneira tanto na presença de um mínimo quanto de um máximo. \\

Por isso, para contornar tais dificuldades, a estratégia proposta por Brent é utilizar uma busca segura (ex: seção-áurea) quando que a função for não cooperativa e trocar para para interpolação parabólica inversa sempre que possível. Assim, incorpora-se a robustez do método da seção-áurea com a rapidez da interpolação quadrática.\\

\begin{figure}[!h]
\centering
    \includegraphics[width=0.75\textwidth]{parabola.pdf}
    \caption{Convergência pelo Método de Brent. Figura retirada de \cite{Brent} }
\label{fig:parabolas}
\end{figure}

A figura \ref{fig:parabolas} mostra duas iterações da interpolação parabólica. É possível notar que a primeira estimativa do minimizador é obtida em $4$ e que, em seguida, esta muda para $5$, se aproximando ainda mais do real valor do minimizador da função $f(x)$.
 
\subsection{Algoritmo}

\vspace{0.5cm}

Durante qualquer fase do algoritmo, este possui um conjunto de 6 pontos $(a, b, u, v, w, x)$. Os pontos $a$ e $b$ definem o intervalo enquadrante de cada iteração; $x$ é o minimizador de $f(x)$ obtido até o momento; $w$ é o ponto de segundo menor valor da função objetivo até o momento e $v$ é o valor anterior de $w$; $u$, por sua vez,  é o ponto onde a função foi avaliada mais recentemente. Um outro ponto que também aparece no algoritmo é o ponto médio entre $a$ e $b$, i.e $x_m$, no entanto, a função objetivo não é avaliada nele.

Se após o passo $d$ da interpolação parabólica, o melhor ponto $x$ pertencer ao intervalo $[a, \, b]$ e, ainda, se o deslocamento provocado por este passo for inferior ao deslocamento provocado na penúltima iteração, significa que $x$ está convergindo e que a interpolação parabólica pode ser usada. Caso alguma dessas condições não seja satisfeita, o passo $d$ é calculado utilizando a razão áurea do maior segmento dentro do intervalo $[a, \,b]$.\\

O algoritmo retoma o início dos cálculos dos novos pontos $(a, b, u, v, w, x)$ até que a tolerância seja atingida ($ |x-x_m| < tol $), alternando,portanto, entre e a interpolação polinomial e o método da razão áurea na determinação dos intervalos para as próximas iterações. O diagrama de fluxo \ref{fig:brentFlow} exemplifica o algoritmo.

 \begin{figure}
 \centering
 \scalebox{1}{% Define block styles
\tikzstyle{io} = [trapezium,trapezium left angle=70,trapezium right angle=-70,minimum height=1cm, draw, fill=yellow!20, text width=6em, text badly centered, node distance=2.5cm, inner sep=0pt]
\tikzstyle{decision} = [diamond, draw, fill=red!20, 
    text width=6em, text badly centered, node distance=3cm, inner sep=0pt]
\tikzstyle{block} = [rectangle, draw, fill=blue!20, 
    text width=10em, text centered, rounded corners, minimum height=4em]
    \tikzstyle{block0} = [rectangle, draw, fill=blue!20, 
    text width=4em, text centered, rounded corners, minimum height=4em]
    \tikzstyle{block1} = [rectangle, draw, fill=cyan!20, node distance=3cm,
    text width=5em, text centered, rounded corners, minimum height=4em]
    \tikzstyle{line} = [draw, -latex']
\tikzstyle{cloud} = [draw, ellipse,fill=green!20, node distance=2cm,
    minimum height=2em]
    \tikzstyle{io2} = [trapezium,trapezium left angle=70,trapezium right angle=-70,minimum height=1cm, draw, fill=yellow!20, text width=1.5cm, text badly centered, node distance=3cm, inner sep=0pt]
    
\begin{tikzpicture}[node distance = 2cm, auto]
    % Place nodes
    \node [io] (input) {$I_1 = [a\, b]$\\$\varepsilon: tolerancia$};
    \node [block, below of =input] (init) {Atualiza $x,w,v,u$\\$x_m = \frac{a+b}{2}$}; % $x=w=v$: criterio 3 pontos\\$d_0=0;
    %\node [block, below of = init] (parab) {Atualiza:\\$x$:minimizador parabola\\$w$:segundo menor valor f\\$v$ : $w$ anterior\\$u$: ultima avaliacao $f$\\};
    \node [block, below of =init] (parab) {$r = (x-w)\,(f_x-f_v)$\\$q = (x-v)\,(f_x-f_w)$\\$p = (x-v)\,q - (x-w)\,r$};
    \node [decision, below of=parab] (evaluate) {$(x+d)\in [a\,b]$\\ $\&\&$\\ $d_k<\frac{d_{k-2}}{2}$};
    \node [below of = evaluate](aux){};
    \node [block1,right of = aux,node distance=3cm] (interp) {$d=\frac{p}{q}$};
    \node [block1, left of=aux, node distance=3cm] (gold) {d: Seção Áurea};
    \node [block0, below of=aux] (act) {$u = x + d$};
    \node [decision, below of=act] (evaluate2) {$|x - x_m|<\varepsilon$};
    \node [io2, below of=evaluate2] (out) {$x_{min}=x$};
    \node [cloud,below of = out] (stop) {FIM};
    %\node [left of = parab_step, node distance=3cm](aux2){};
       
    % Draw edges
    \path [line] (input) -- (init);
    \path [line] (init) -- (parab);
    \path [line] (parab) -- (evaluate);
    \path [line,rounded corners] (evaluate.east) -| node [above] {sim} (interp);
    \path [line,rounded corners] (evaluate.west) -| node [above] {não}(gold);
    \path [line,rounded corners] (gold.south) |- (act.west);
    \path [line,rounded corners] (interp.south) |- (act.east);
    \path[line] (act) -- (evaluate2);
    \path [line] (evaluate2) -- node [near start] {sim} (out);
    \path [line,rounded corners] (evaluate2.east)-- ++(35mm,0mm) |- node [above] {não} (init.east);
    \path [line] (out) -- (stop);
\end{tikzpicture}}
 \caption{Fluxograma do Método de Interpolação}
 \label{fig:brentFlow}
 \end{figure}

\subsection{Resultado}

O resultado obtido para o método de Brent utilizando a função teste apresentada previamente neste relatório (eq. \ref{eq:func_obj}) pode ser observado na figura \ref{fig:brent}.\\

\begin{figure}[h!]
\centering
\scalebox{1}{\section{Interpolação Parabólica e o Método de Brent}

\vspace{0.5cm}

\subsection{Princípios básicos}

\vspace{0.5cm}

A idéia por trás desse método é utilizar uma aproximação da função $f(x)$ por um polinômio de grau 2 $g(x)$ que passa por três pontos pertencentes a um intervalo enquadrante. Dessa maneira, aproxima-se o mínimo de $f(x)$ pelo mínimo da parábola obtida. A fórmula \ref{eq:parab} permite encontrar a abcissa $x$ correspondente ao mínimo da parábola para os pontos $a,b$ e $c$.\\

\begin{equation}
  x = b - \frac{1}{2}\frac{(b-a)^2[f(b)-f(c)] - (b-c)^2[f(b)-f(a)]}{(b-a)[f(b)-f(c)] - (b-c)[f(b)-f(a)]} 
  \label{eq:parab}
\end{equation}

\vspace{0.5cm}

Redefine-se, então, o intervalo para a próxima iteração de acordo com a avaliação da função objetivo nesses três pontos, descartando o pior deles e guardando o minimizador desta parábola aproximada como novo ponto. Tal interpolação parabólica é dita inversa pois buscamos a abscissa $x$ e não a ordenada $f(x)$. \\

Da expressão \ref{eq:parab}, podemos notar que existe uma singularidade quando os três pontos são colineares. Além disso, o método funciona da mesma maneira tanto na presença de um mínimo quanto de um máximo. \\

Por isso, para contornar tais dificuldades, a estratégia proposta por Brent é utilizar uma busca segura (ex: seção-áurea) quando que a função for não cooperativa e trocar para para interpolação parabólica inversa sempre que possível. Assim, incorpora-se a robustez do método da seção-áurea com a rapidez da interpolação quadrática.\\

\begin{figure}[!h]
\centering
    \includegraphics[width=0.75\textwidth]{parabola.pdf}
    \caption{Convergência pelo Método de Brent. Figura retirada de \cite{Brent} }
\label{fig:parabolas}
\end{figure}

A figura \ref{fig:parabolas} mostra duas iterações da interpolação parabólica. É possível notar que a primeira estimativa do minimizador é obtida em $4$ e que, em seguida, esta muda para $5$, se aproximando ainda mais do real valor do minimizador da função $f(x)$.
 
\subsection{Algoritmo}

\vspace{0.5cm}

Durante qualquer fase do algoritmo, este possui um conjunto de 6 pontos $(a, b, u, v, w, x)$. Os pontos $a$ e $b$ definem o intervalo enquadrante de cada iteração; $x$ é o minimizador de $f(x)$ obtido até o momento; $w$ é o ponto de segundo menor valor da função objetivo até o momento e $v$ é o valor anterior de $w$; $u$, por sua vez,  é o ponto onde a função foi avaliada mais recentemente. Um outro ponto que também aparece no algoritmo é o ponto médio entre $a$ e $b$, i.e $x_m$, no entanto, a função objetivo não é avaliada nele.

Se após o passo $d$ da interpolação parabólica, o melhor ponto $x$ pertencer ao intervalo $[a, \, b]$ e, ainda, se o deslocamento provocado por este passo for inferior ao deslocamento provocado na penúltima iteração, significa que $x$ está convergindo e que a interpolação parabólica pode ser usada. Caso alguma dessas condições não seja satisfeita, o passo $d$ é calculado utilizando a razão áurea do maior segmento dentro do intervalo $[a, \,b]$.\\

O algoritmo retoma o início dos cálculos dos novos pontos $(a, b, u, v, w, x)$ até que a tolerância seja atingida ($ |x-x_m| < tol $), alternando,portanto, entre e a interpolação polinomial e o método da razão áurea na determinação dos intervalos para as próximas iterações. O diagrama de fluxo \ref{fig:brentFlow} exemplifica o algoritmo.

 \begin{figure}
 \centering
 \scalebox{1}{\input{brentFlow.tex}}
 \caption{Fluxograma do Método de Interpolação}
 \label{fig:brentFlow}
 \end{figure}

\subsection{Resultado}

O resultado obtido para o método de Brent utilizando a função teste apresentada previamente neste relatório (eq. \ref{eq:func_obj}) pode ser observado na figura \ref{fig:brent}.\\

\begin{figure}[h!]
\centering
\scalebox{1}{\input{./figuras/brent.tex}}
\caption{Tolerância: $\epsilon =  0.01$. Número de iterações: $n = 12$}
\label{fig:brent}
\end{figure}

É interessante notar que a performance foi bem similar ao método da seção áurea já que a tolerância $\epsilon$ escolhida foi a mesma. No entanto, o método de Brent é consideravelmente mais rápido devido à utilizão da interpolação polinomial. Ao passo que foram necessárias 17 iterações para Seção Áurea convergir, somente 12 bastaram para garantir o mesmo resultado usando o método de Brent.}
\caption{Tolerância: $\epsilon =  0.01$. Número de iterações: $n = 12$}
\label{fig:brent}
\end{figure}

É interessante notar que a performance foi bem similar ao método da seção áurea já que a tolerância $\epsilon$ escolhida foi a mesma. No entanto, o método de Brent é consideravelmente mais rápido devido à utilizão da interpolação polinomial. Ao passo que foram necessárias 17 iterações para Seção Áurea convergir, somente 12 bastaram para garantir o mesmo resultado usando o método de Brent.}
\caption{Tolerância: $\epsilon =  0.01$. Número de iterações: $n = 12$}
\label{fig:brent}
\end{figure}

É interessante notar que a performance foi bem similar ao método da seção áurea já que a tolerância $\epsilon$ escolhida foi a mesma. No entanto, o método de Brent é consideravelmente mais rápido devido à utilizão da interpolação polinomial. Ao passo que foram necessárias 17 iterações para Seção Áurea convergir, somente 12 bastaram para garantir o mesmo resultado usando o método de Brent.}
\caption{Tolerância: $\epsilon =  0.01$. Número de iterações: $n = 12$}
\label{fig:brent}
\end{figure}

É interessante notar que a performance foi bem similar ao método da seção áurea já que a tolerância $\epsilon$ escolhida foi a mesma. No entanto, o método de Brent é consideravelmente mais rápido devido à utilizão da interpolação polinomial. Ao passo que foram necessárias 17 iterações para Seção Áurea convergir, somente 12 bastaram para garantir o mesmo resultado usando o método de Brent.

% \section{Interpolação Parabólica e o Método de Brent}

\vspace{0.5cm}

\subsection{Princípios básicos}

\vspace{0.5cm}

A idéia por trás desse método é utilizar uma aproximação da função $f(x)$ por um polinômio de grau 2 $g(x)$ que passa por três pontos pertencentes a um intervalo enquadrante. Dessa maneira, aproxima-se o mínimo de $f(x)$ pelo mínimo da parábola obtida. A fórmula \ref{eq:parab} permite encontrar a abcissa $x$ correspondente ao mínimo da parábola para os pontos $a,b$ e $c$.\\

\begin{equation}
  x = b - \frac{1}{2}\frac{(b-a)^2[f(b)-f(c)] - (b-c)^2[f(b)-f(a)]}{(b-a)[f(b)-f(c)] - (b-c)[f(b)-f(a)]} 
  \label{eq:parab}
\end{equation}

\vspace{0.5cm}

Redefine-se, então, o intervalo para a próxima iteração de acordo com a avaliação da função objetivo nesses três pontos, descartando o pior deles e guardando o minimizador desta parábola aproximada como novo ponto. Tal interpolação parabólica é dita inversa pois buscamos a abscissa $x$ e não a ordenada $f(x)$. \\

Da expressão \ref{eq:parab}, podemos notar que existe uma singularidade quando os três pontos são colineares. Além disso, o método funciona da mesma maneira tanto na presença de um mínimo quanto de um máximo. \\

Por isso, para contornar tais dificuldades, a estratégia proposta por Brent é utilizar uma busca segura (ex: seção-áurea) quando que a função for não cooperativa e trocar para para interpolação parabólica inversa sempre que possível. Assim, incorpora-se a robustez do método da seção-áurea com a rapidez da interpolação quadrática.\\

\begin{figure}[!h]
\centering
    \includegraphics[width=0.75\textwidth]{parabola.pdf}
    \caption{Convergência pelo Método de Brent. Figura retirada de \cite{Brent} }
\label{fig:parabolas}
\end{figure}

A figura \ref{fig:parabolas} mostra duas iterações da interpolação parabólica. É possível notar que a primeira estimativa do minimizador é obtida em $4$ e que, em seguida, esta muda para $5$, se aproximando ainda mais do real valor do minimizador da função $f(x)$.
 
\subsection{Algoritmo}

\vspace{0.5cm}

Durante qualquer fase do algoritmo, este possui um conjunto de 6 pontos $(a, b, u, v, w, x)$. Os pontos $a$ e $b$ definem o intervalo enquadrante de cada iteração; $x$ é o minimizador de $f(x)$ obtido até o momento; $w$ é o ponto de segundo menor valor da função objetivo até o momento e $v$ é o valor anterior de $w$; $u$, por sua vez,  é o ponto onde a função foi avaliada mais recentemente. Um outro ponto que também aparece no algoritmo é o ponto médio entre $a$ e $b$, i.e $x_m$, no entanto, a função objetivo não é avaliada nele.

Se após o passo $d$ da interpolação parabólica, o melhor ponto $x$ pertencer ao intervalo $[a, \, b]$ e, ainda, se o deslocamento provocado por este passo for inferior ao deslocamento provocado na penúltima iteração, significa que $x$ está convergindo e que a interpolação parabólica pode ser usada. Caso alguma dessas condições não seja satisfeita, o passo $d$ é calculado utilizando a razão áurea do maior segmento dentro do intervalo $[a, \,b]$.\\

O algoritmo retoma o início dos cálculos dos novos pontos $(a, b, u, v, w, x)$ até que a tolerância seja atingida ($ |x-x_m| < tol $), alternando,portanto, entre e a interpolação polinomial e o método da razão áurea na determinação dos intervalos para as próximas iterações. O diagrama de fluxo \ref{fig:brentFlow} exemplifica o algoritmo.

 \begin{figure}
 \centering
 \scalebox{1}{% Define block styles
\tikzstyle{io} = [trapezium,trapezium left angle=70,trapezium right angle=-70,minimum height=1cm, draw, fill=yellow!20, text width=6em, text badly centered, node distance=2.5cm, inner sep=0pt]
\tikzstyle{decision} = [diamond, draw, fill=red!20, 
    text width=6em, text badly centered, node distance=3cm, inner sep=0pt]
\tikzstyle{block} = [rectangle, draw, fill=blue!20, 
    text width=10em, text centered, rounded corners, minimum height=4em]
    \tikzstyle{block0} = [rectangle, draw, fill=blue!20, 
    text width=4em, text centered, rounded corners, minimum height=4em]
    \tikzstyle{block1} = [rectangle, draw, fill=cyan!20, node distance=3cm,
    text width=5em, text centered, rounded corners, minimum height=4em]
    \tikzstyle{line} = [draw, -latex']
\tikzstyle{cloud} = [draw, ellipse,fill=green!20, node distance=2cm,
    minimum height=2em]
    \tikzstyle{io2} = [trapezium,trapezium left angle=70,trapezium right angle=-70,minimum height=1cm, draw, fill=yellow!20, text width=1.5cm, text badly centered, node distance=3cm, inner sep=0pt]
    
\begin{tikzpicture}[node distance = 2cm, auto]
    % Place nodes
    \node [io] (input) {$I_1 = [a\, b]$\\$\varepsilon: tolerancia$};
    \node [block, below of =input] (init) {Atualiza $x,w,v,u$\\$x_m = \frac{a+b}{2}$}; % $x=w=v$: criterio 3 pontos\\$d_0=0;
    %\node [block, below of = init] (parab) {Atualiza:\\$x$:minimizador parabola\\$w$:segundo menor valor f\\$v$ : $w$ anterior\\$u$: ultima avaliacao $f$\\};
    \node [block, below of =init] (parab) {$r = (x-w)\,(f_x-f_v)$\\$q = (x-v)\,(f_x-f_w)$\\$p = (x-v)\,q - (x-w)\,r$};
    \node [decision, below of=parab] (evaluate) {$(x+d)\in [a\,b]$\\ $\&\&$\\ $d_k<\frac{d_{k-2}}{2}$};
    \node [below of = evaluate](aux){};
    \node [block1,right of = aux,node distance=3cm] (interp) {$d=\frac{p}{q}$};
    \node [block1, left of=aux, node distance=3cm] (gold) {d: Seção Áurea};
    \node [block0, below of=aux] (act) {$u = x + d$};
    \node [decision, below of=act] (evaluate2) {$|x - x_m|<\varepsilon$};
    \node [io2, below of=evaluate2] (out) {$x_{min}=x$};
    \node [cloud,below of = out] (stop) {FIM};
    %\node [left of = parab_step, node distance=3cm](aux2){};
       
    % Draw edges
    \path [line] (input) -- (init);
    \path [line] (init) -- (parab);
    \path [line] (parab) -- (evaluate);
    \path [line,rounded corners] (evaluate.east) -| node [above] {sim} (interp);
    \path [line,rounded corners] (evaluate.west) -| node [above] {não}(gold);
    \path [line,rounded corners] (gold.south) |- (act.west);
    \path [line,rounded corners] (interp.south) |- (act.east);
    \path[line] (act) -- (evaluate2);
    \path [line] (evaluate2) -- node [near start] {sim} (out);
    \path [line,rounded corners] (evaluate2.east)-- ++(35mm,0mm) |- node [above] {não} (init.east);
    \path [line] (out) -- (stop);
\end{tikzpicture}}
 \caption{Fluxograma do Método de Interpolação}
 \label{fig:brentFlow}
 \end{figure}

\subsection{Resultado}

O resultado obtido para o método de Brent utilizando a função teste apresentada previamente neste relatório (eq. \ref{eq:func_obj}) pode ser observado na figura \ref{fig:brent}.\\

\begin{figure}[h!]
\centering
\scalebox{1}{\section{Interpolação Parabólica e o Método de Brent}

\vspace{0.5cm}

\subsection{Princípios básicos}

\vspace{0.5cm}

A idéia por trás desse método é utilizar uma aproximação da função $f(x)$ por um polinômio de grau 2 $g(x)$ que passa por três pontos pertencentes a um intervalo enquadrante. Dessa maneira, aproxima-se o mínimo de $f(x)$ pelo mínimo da parábola obtida. A fórmula \ref{eq:parab} permite encontrar a abcissa $x$ correspondente ao mínimo da parábola para os pontos $a,b$ e $c$.\\

\begin{equation}
  x = b - \frac{1}{2}\frac{(b-a)^2[f(b)-f(c)] - (b-c)^2[f(b)-f(a)]}{(b-a)[f(b)-f(c)] - (b-c)[f(b)-f(a)]} 
  \label{eq:parab}
\end{equation}

\vspace{0.5cm}

Redefine-se, então, o intervalo para a próxima iteração de acordo com a avaliação da função objetivo nesses três pontos, descartando o pior deles e guardando o minimizador desta parábola aproximada como novo ponto. Tal interpolação parabólica é dita inversa pois buscamos a abscissa $x$ e não a ordenada $f(x)$. \\

Da expressão \ref{eq:parab}, podemos notar que existe uma singularidade quando os três pontos são colineares. Além disso, o método funciona da mesma maneira tanto na presença de um mínimo quanto de um máximo. \\

Por isso, para contornar tais dificuldades, a estratégia proposta por Brent é utilizar uma busca segura (ex: seção-áurea) quando que a função for não cooperativa e trocar para para interpolação parabólica inversa sempre que possível. Assim, incorpora-se a robustez do método da seção-áurea com a rapidez da interpolação quadrática.\\

\begin{figure}[!h]
\centering
    \includegraphics[width=0.75\textwidth]{parabola.pdf}
    \caption{Convergência pelo Método de Brent. Figura retirada de \cite{Brent} }
\label{fig:parabolas}
\end{figure}

A figura \ref{fig:parabolas} mostra duas iterações da interpolação parabólica. É possível notar que a primeira estimativa do minimizador é obtida em $4$ e que, em seguida, esta muda para $5$, se aproximando ainda mais do real valor do minimizador da função $f(x)$.
 
\subsection{Algoritmo}

\vspace{0.5cm}

Durante qualquer fase do algoritmo, este possui um conjunto de 6 pontos $(a, b, u, v, w, x)$. Os pontos $a$ e $b$ definem o intervalo enquadrante de cada iteração; $x$ é o minimizador de $f(x)$ obtido até o momento; $w$ é o ponto de segundo menor valor da função objetivo até o momento e $v$ é o valor anterior de $w$; $u$, por sua vez,  é o ponto onde a função foi avaliada mais recentemente. Um outro ponto que também aparece no algoritmo é o ponto médio entre $a$ e $b$, i.e $x_m$, no entanto, a função objetivo não é avaliada nele.

Se após o passo $d$ da interpolação parabólica, o melhor ponto $x$ pertencer ao intervalo $[a, \, b]$ e, ainda, se o deslocamento provocado por este passo for inferior ao deslocamento provocado na penúltima iteração, significa que $x$ está convergindo e que a interpolação parabólica pode ser usada. Caso alguma dessas condições não seja satisfeita, o passo $d$ é calculado utilizando a razão áurea do maior segmento dentro do intervalo $[a, \,b]$.\\

O algoritmo retoma o início dos cálculos dos novos pontos $(a, b, u, v, w, x)$ até que a tolerância seja atingida ($ |x-x_m| < tol $), alternando,portanto, entre e a interpolação polinomial e o método da razão áurea na determinação dos intervalos para as próximas iterações. O diagrama de fluxo \ref{fig:brentFlow} exemplifica o algoritmo.

 \begin{figure}
 \centering
 \scalebox{1}{% Define block styles
\tikzstyle{io} = [trapezium,trapezium left angle=70,trapezium right angle=-70,minimum height=1cm, draw, fill=yellow!20, text width=6em, text badly centered, node distance=2.5cm, inner sep=0pt]
\tikzstyle{decision} = [diamond, draw, fill=red!20, 
    text width=6em, text badly centered, node distance=3cm, inner sep=0pt]
\tikzstyle{block} = [rectangle, draw, fill=blue!20, 
    text width=10em, text centered, rounded corners, minimum height=4em]
    \tikzstyle{block0} = [rectangle, draw, fill=blue!20, 
    text width=4em, text centered, rounded corners, minimum height=4em]
    \tikzstyle{block1} = [rectangle, draw, fill=cyan!20, node distance=3cm,
    text width=5em, text centered, rounded corners, minimum height=4em]
    \tikzstyle{line} = [draw, -latex']
\tikzstyle{cloud} = [draw, ellipse,fill=green!20, node distance=2cm,
    minimum height=2em]
    \tikzstyle{io2} = [trapezium,trapezium left angle=70,trapezium right angle=-70,minimum height=1cm, draw, fill=yellow!20, text width=1.5cm, text badly centered, node distance=3cm, inner sep=0pt]
    
\begin{tikzpicture}[node distance = 2cm, auto]
    % Place nodes
    \node [io] (input) {$I_1 = [a\, b]$\\$\varepsilon: tolerancia$};
    \node [block, below of =input] (init) {Atualiza $x,w,v,u$\\$x_m = \frac{a+b}{2}$}; % $x=w=v$: criterio 3 pontos\\$d_0=0;
    %\node [block, below of = init] (parab) {Atualiza:\\$x$:minimizador parabola\\$w$:segundo menor valor f\\$v$ : $w$ anterior\\$u$: ultima avaliacao $f$\\};
    \node [block, below of =init] (parab) {$r = (x-w)\,(f_x-f_v)$\\$q = (x-v)\,(f_x-f_w)$\\$p = (x-v)\,q - (x-w)\,r$};
    \node [decision, below of=parab] (evaluate) {$(x+d)\in [a\,b]$\\ $\&\&$\\ $d_k<\frac{d_{k-2}}{2}$};
    \node [below of = evaluate](aux){};
    \node [block1,right of = aux,node distance=3cm] (interp) {$d=\frac{p}{q}$};
    \node [block1, left of=aux, node distance=3cm] (gold) {d: Seção Áurea};
    \node [block0, below of=aux] (act) {$u = x + d$};
    \node [decision, below of=act] (evaluate2) {$|x - x_m|<\varepsilon$};
    \node [io2, below of=evaluate2] (out) {$x_{min}=x$};
    \node [cloud,below of = out] (stop) {FIM};
    %\node [left of = parab_step, node distance=3cm](aux2){};
       
    % Draw edges
    \path [line] (input) -- (init);
    \path [line] (init) -- (parab);
    \path [line] (parab) -- (evaluate);
    \path [line,rounded corners] (evaluate.east) -| node [above] {sim} (interp);
    \path [line,rounded corners] (evaluate.west) -| node [above] {não}(gold);
    \path [line,rounded corners] (gold.south) |- (act.west);
    \path [line,rounded corners] (interp.south) |- (act.east);
    \path[line] (act) -- (evaluate2);
    \path [line] (evaluate2) -- node [near start] {sim} (out);
    \path [line,rounded corners] (evaluate2.east)-- ++(35mm,0mm) |- node [above] {não} (init.east);
    \path [line] (out) -- (stop);
\end{tikzpicture}}
 \caption{Fluxograma do Método de Interpolação}
 \label{fig:brentFlow}
 \end{figure}

\subsection{Resultado}

O resultado obtido para o método de Brent utilizando a função teste apresentada previamente neste relatório (eq. \ref{eq:func_obj}) pode ser observado na figura \ref{fig:brent}.\\

\begin{figure}[h!]
\centering
\scalebox{1}{\section{Interpolação Parabólica e o Método de Brent}

\vspace{0.5cm}

\subsection{Princípios básicos}

\vspace{0.5cm}

A idéia por trás desse método é utilizar uma aproximação da função $f(x)$ por um polinômio de grau 2 $g(x)$ que passa por três pontos pertencentes a um intervalo enquadrante. Dessa maneira, aproxima-se o mínimo de $f(x)$ pelo mínimo da parábola obtida. A fórmula \ref{eq:parab} permite encontrar a abcissa $x$ correspondente ao mínimo da parábola para os pontos $a,b$ e $c$.\\

\begin{equation}
  x = b - \frac{1}{2}\frac{(b-a)^2[f(b)-f(c)] - (b-c)^2[f(b)-f(a)]}{(b-a)[f(b)-f(c)] - (b-c)[f(b)-f(a)]} 
  \label{eq:parab}
\end{equation}

\vspace{0.5cm}

Redefine-se, então, o intervalo para a próxima iteração de acordo com a avaliação da função objetivo nesses três pontos, descartando o pior deles e guardando o minimizador desta parábola aproximada como novo ponto. Tal interpolação parabólica é dita inversa pois buscamos a abscissa $x$ e não a ordenada $f(x)$. \\

Da expressão \ref{eq:parab}, podemos notar que existe uma singularidade quando os três pontos são colineares. Além disso, o método funciona da mesma maneira tanto na presença de um mínimo quanto de um máximo. \\

Por isso, para contornar tais dificuldades, a estratégia proposta por Brent é utilizar uma busca segura (ex: seção-áurea) quando que a função for não cooperativa e trocar para para interpolação parabólica inversa sempre que possível. Assim, incorpora-se a robustez do método da seção-áurea com a rapidez da interpolação quadrática.\\

\begin{figure}[!h]
\centering
    \includegraphics[width=0.75\textwidth]{parabola.pdf}
    \caption{Convergência pelo Método de Brent. Figura retirada de \cite{Brent} }
\label{fig:parabolas}
\end{figure}

A figura \ref{fig:parabolas} mostra duas iterações da interpolação parabólica. É possível notar que a primeira estimativa do minimizador é obtida em $4$ e que, em seguida, esta muda para $5$, se aproximando ainda mais do real valor do minimizador da função $f(x)$.
 
\subsection{Algoritmo}

\vspace{0.5cm}

Durante qualquer fase do algoritmo, este possui um conjunto de 6 pontos $(a, b, u, v, w, x)$. Os pontos $a$ e $b$ definem o intervalo enquadrante de cada iteração; $x$ é o minimizador de $f(x)$ obtido até o momento; $w$ é o ponto de segundo menor valor da função objetivo até o momento e $v$ é o valor anterior de $w$; $u$, por sua vez,  é o ponto onde a função foi avaliada mais recentemente. Um outro ponto que também aparece no algoritmo é o ponto médio entre $a$ e $b$, i.e $x_m$, no entanto, a função objetivo não é avaliada nele.

Se após o passo $d$ da interpolação parabólica, o melhor ponto $x$ pertencer ao intervalo $[a, \, b]$ e, ainda, se o deslocamento provocado por este passo for inferior ao deslocamento provocado na penúltima iteração, significa que $x$ está convergindo e que a interpolação parabólica pode ser usada. Caso alguma dessas condições não seja satisfeita, o passo $d$ é calculado utilizando a razão áurea do maior segmento dentro do intervalo $[a, \,b]$.\\

O algoritmo retoma o início dos cálculos dos novos pontos $(a, b, u, v, w, x)$ até que a tolerância seja atingida ($ |x-x_m| < tol $), alternando,portanto, entre e a interpolação polinomial e o método da razão áurea na determinação dos intervalos para as próximas iterações. O diagrama de fluxo \ref{fig:brentFlow} exemplifica o algoritmo.

 \begin{figure}
 \centering
 \scalebox{1}{\input{brentFlow.tex}}
 \caption{Fluxograma do Método de Interpolação}
 \label{fig:brentFlow}
 \end{figure}

\subsection{Resultado}

O resultado obtido para o método de Brent utilizando a função teste apresentada previamente neste relatório (eq. \ref{eq:func_obj}) pode ser observado na figura \ref{fig:brent}.\\

\begin{figure}[h!]
\centering
\scalebox{1}{\input{./figuras/brent.tex}}
\caption{Tolerância: $\epsilon =  0.01$. Número de iterações: $n = 12$}
\label{fig:brent}
\end{figure}

É interessante notar que a performance foi bem similar ao método da seção áurea já que a tolerância $\epsilon$ escolhida foi a mesma. No entanto, o método de Brent é consideravelmente mais rápido devido à utilizão da interpolação polinomial. Ao passo que foram necessárias 17 iterações para Seção Áurea convergir, somente 12 bastaram para garantir o mesmo resultado usando o método de Brent.}
\caption{Tolerância: $\epsilon =  0.01$. Número de iterações: $n = 12$}
\label{fig:brent}
\end{figure}

É interessante notar que a performance foi bem similar ao método da seção áurea já que a tolerância $\epsilon$ escolhida foi a mesma. No entanto, o método de Brent é consideravelmente mais rápido devido à utilizão da interpolação polinomial. Ao passo que foram necessárias 17 iterações para Seção Áurea convergir, somente 12 bastaram para garantir o mesmo resultado usando o método de Brent.}
\caption{Tolerância: $\epsilon =  0.01$. Número de iterações: $n = 12$}
\label{fig:brent}
\end{figure}

É interessante notar que a performance foi bem similar ao método da seção áurea já que a tolerância $\epsilon$ escolhida foi a mesma. No entanto, o método de Brent é consideravelmente mais rápido devido à utilizão da interpolação polinomial. Ao passo que foram necessárias 17 iterações para Seção Áurea convergir, somente 12 bastaram para garantir o mesmo resultado usando o método de Brent.

\bibliography{bibliografia}
\bibliographystyle{plain}


\end{document}
